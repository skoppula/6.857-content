%
% 6.857 homework template
%
% NOTE:
% Be sure to define your team members with the \team command
% Be sure to define the problem set with the \ps command
% Be sure to use the \answer command for each of your answers 
\documentclass[11pt]{article}

\newcommand{\team}{ Erica Du \\ Skanda Koppula \\ Jessica Wang }
\newcommand{\ps}{ Problem Set 1 }

%\pagestyle{headings}
\usepackage{amsfonts}
\usepackage{minted}
\usepackage{graphicx}
\usepackage{amssymb}
\usepackage{amsmath}
\usepackage{latexsym}
\setlength{\parskip}{1pc}
\setlength{\parindent}{0pt}
\setlength{\topmargin}{-3pc}
\setlength{\textheight}{9.5in}
\setlength{\oddsidemargin}{0pc}
\setlength{\evensidemargin}{0pc}
\setlength{\textwidth}{6.5in}

\newcommand{\answer}[1]{
\newpage
\noindent
\framebox{
	\vbox{
		6.857 Homework \hfill {\bf \ps} \hfill \# #1  \\ 
		\team \hfill \today
	}
}
\bigskip

}


\begin{document}

\answer{1-1 - An edX Security Policy}

\textbf{Objective}: The EdX MOOC platform aims to provide educational content to users of the platform. The platform includes other auxiliary objectives such as being able to update course content, provide inter-person interactivity, and reward users for course completion.\\

\textbf{Principals and Authorized Behavior}: The platform has three main roles: student, teacher, and administrator. Each have their own set of allowed functions:
\begin{itemize}
\item Student: a student must be able to request enrollment into a course, view content for an  enrolled courses (to the extent allowed by course instructor(s)), submit digitally signed answers and feedback to (enrolled) course instructor, modify their own profile information (login username, password, email), and receive certificate of completions as determined by the instructor.
\indent\item Honor Code Certificate Candidate
\indent\item Verified Certificate Candidate
\indent Course Staff (Instructors and TAs): authorize student enrollment into their course, authorize other instructors to co-administer their course, modify course meta-data (name, description) and content, send signed feedback to student, view the submissions of student enrolled in their course(s), modify their own profile information (email, login username, password), add new course, and award certificate of completion to select student users.
\item Universities:
\item Everyone
\item edX Employees
\end{itemize}

\textbf{Confidentiality/Integrity Details:}: Specifically, by use of some security mechanism (perhaps, OAuth 2.0 or like), we intend that the platform forbids the student viewing or modify the content and progress of other users (except to view the content of their enrolled courses as allowed by their course instructors). The student should not be able to obtain completion certificates without proper authentication.\\

The teacher should not be able to view the submissions of students not currently in their course. They should not be able to update the content of the courses that they were not authorized to administer.\\

Out of the 3 security goals, integrity is the most relevant for edX. 
\item having students changing other students' data
\item course staff changing other course staff's courses
\item students changing course staff's data
\item having students earn their certificates honorably

Confidentiality:
\item only students and teachers should be able to see grades

Availability:
\item not as pressing as say, a hospital service, not endangering lives
\item everything/permissions should be maintained even if service becomes unavailable

\answer{1-2 - Reused Pad Cipher}
A. They were encrypted using the same pad. The two words are ADVERSARIAL AND MATHEMATICS. We used the following script:
\inputminted{python}{onepad.py}

B.
\item iterate between all pairs of ciphertexts
\item xor the pair together
\item search through xor'd output and try to find common words
\item if found, we have a winner
\item else, move on to next pair

\answer{1-3 - Reused Pad Cipher}
\begin{itemize}
  \item Insecure implementation - The implementation of a theoretically secure algorithms might have vulnerabilities. For example, the OpenSSL library this year was revealed to have a buffer over-read bug, Heartbleed.
  \item Improper use -  users might misuse otherwise secure software/algorithms. For example, short/weak keys and re-using passwords could introduce vulnerabilities.
  \item Legal retrieval - the government (for example) might file court orders to obtain private keys
  \item Social engineering - adversaries could extract personal information. A great example might be email or phone call phishing, in which unsuspecting victims reveal personal information to agents posing as authorities.
  \item Hardware injection - an example is keyloggers, tracking keystrokes and potentially passwords, or inserting things into routers.
  \item backdoors, bribery, coercion
  \item Using metadata or indirect data such as timing (i.e. side channel timing attacks)
  \item Insert faulty code into commonly used encryption tools/algorithms, like random number generators.
\end{itemize}


\answer{1-3 - Vulnerability/Mechanism Chains}



\end{document}

