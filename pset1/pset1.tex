%
% 6.857 homework template
%
% NOTE:
% Be sure to define your team members with the \team command
% Be sure to define the problem set with the \ps command
% Be sure to use the \answer command for each of your answers 
\documentclass[11pt]{article}

\newcommand{\team}{ Erica Du \\ Skanda Koppula \\ Jessica Wang }
\newcommand{\ps}{ Problem Set 1 }

%\pagestyle{headings}
\usepackage{amsfonts}
\usepackage{minted}
\usepackage{graphicx}
\usepackage{amssymb}
\usepackage{amsmath}
\usepackage{latexsym}
\setlength{\parskip}{1pc}
\setlength{\parindent}{0pt}
\setlength{\topmargin}{-3pc}
\setlength{\textheight}{9.5in}
\setlength{\oddsidemargin}{0pc}
\setlength{\evensidemargin}{0pc}
\setlength{\textwidth}{6.5in}

\newcommand{\answer}[1]{
\newpage
\noindent
\framebox{
	\vbox{
		6.857 Homework \hfill {\bf \ps} \hfill \# #1  \\ 
		\team \hfill \today
	}
}
\bigskip

}


\begin{document}

\answer{1-1 - An edX Security Policy}

\section{Objective}
The purpose of this security policy is to outline the security goals and usages of a massive open online course platform (MOOC) like edX. In this document, we define policies regarding user roles and their permissions, as well as privacy and sharing of data.

\section{Introduction}
EdX is a non-profit, open source MOOC platform that aims to provide educational content to worldwide users. Users sign up on edX and gain access to online courses in a wide range of subjects, administered by accredited universities and organizations around the world. After successful completion of a course, a user may achieve a certificate of completion.\\

Courses, run by universities, are administered by staff that handle the operations of the course. These staff members control course content and the student experience.\\

\section{Principals and Authorized Behavior}
The platform has a hierarchy of roles. They are, in order: edX employees, universities, course staff, students, and everyone. Each have their own set of allowed functions:

\subsection{edX employees}
\subsection{Universities}
\subsection{Course staff}
Course staff comprises of both Instructors and TAs. Course staff must be able to authorize student enrollment into their course, authorize other instructors to co-administer their course, modify course meta-data (name, description) and content, send signed feedback to student, view the submissions of student enrolled in their course(s), modify their own profile information (email, login username, password), add new courses, and award certificate of completion to select student users.
\subsection{Students}
When logged in, a student must be able to request enrollment into a course, view content for an  enrolled courses (to the extent allowed by course instructor(s)), submit digitally signed answers and feedback to (enrolled) course instructor, modify their own profile information (login username, password, email), and receive certificate of completions as determined by the instructor.\\

Honor Code Certificate Candidate\\
Verified Certificate Candidate\\

\subsection{Everyone}
Everyone should be able to view an introductory video, read the course description, and read short bios of the instructors of every edX course. Everyone must also be able to register for a new account, or sign in to an existing account.

\section{Authentication}
\begin{itemize}
\item edX password-protects all user accounts. These passwords are encrypted and salted. 
\item Alternatively, edX also handles authentication through users' Facebook or Google+ accounts.
\item Currently the edX platform supports Shibboleth, CAS, and SSL certificates as external authentication sources.
\end{itemize}

\section{Security Goals}
Out of the 3 general security goals, integrity is the most relevant and crucial for edX. Since edX-type platforms are metaphorical to a traditional classroom, courses need to maintain the integrity of their data, and unauthorized users such as students or administrators of other courses should not be able to modify a specific course's data. Confidentiality is another important goal - the information of edX users should not be compromised to unauthorized parties.

\subsection{Integrity}
Specific security goals that fall under integrity include the following.
\begin{itemize}
\item Ensuring that anyone with permissions of or lower than that of a student cannot change another student's data or course data.
\item Ensuring that course staff cannot change data of courses they are not administrators of.
\item Ensuring that students cannot change their own data in an abnormal way, e.g. unauthorized modification of test scores.
\item Ensuring that students earn completion certificates honorably.
\end{itemize}

\subsection{Confidentiality}
\begin{itemize}
\item only students and teachers should be able to see grades
\end{itemize}

\subsection{Availability}
\begin{itemize}
\item not as pressing as say, a hospital service, not endangering lives
\item everything/permissions should be maintained even if service becomes unavailable
\end{itemize}

\textbf{Confidentiality/Integrity Details:}: Specifically, by use of some security mechanism (perhaps, OAuth 2.0 or like), we intend that the platform forbids the student viewing or modify the content and progress of other users (except to view the content of their enrolled courses as allowed by their course instructors). The student should not be able to obtain completion certificates without proper authentication.\\

The teacher should not be able to view the submissions of students not currently in their course. They should not be able to update the content of the courses that they were not authorized to administer.\\

\answer{1-2 - Reused Pad Cipher}
A. They were encrypted using the same pad. The two words are ADVERSARIAL AND MATHEMATICS. We used the following script:
\inputminted{python}{2a.py}

B. There are four messages and three unique pads. Therefore at least two messages will have the same two pads applied to them ($3 \choose 2$$=3$). If we XOR those two pads together, the pads will cancel. Once we obtain the pad-canceled XOR of two messages, we peel apart the two messages via crib-dragging.\\

First, to identify which of the six possible pairwise combinations have the pads cancel out, we compute all six and count the frequency of '0' in the hex representation of each cross. All lowercase ascii [a-z] share the same last four bits, so the XOR of two plaintext paragraphs will contain a signature with plenty of zeroes.\\

		Now we try peeling apart the two messages via crib-dragging. We took a few common words like ' the ' or ' is ' and XOR'd them along every position in our message product. If the common word is in that position in one of the messages, bits of the other message will be revealed. We extrapolate and guess words from those bits, and repeat the process. \mintinline{bash}{grep} us helped extrapolate possibilities for words given the dictionary. We were also thinking of XOR'ing big words and testing for resulting smaller words in the other message.\\

We wrote the following interactive script to speed up and automate entering a crib, XOR'ing along the positions, and choosing the correct position (if any).

\inputminted{python}{2b.py}

From this we discovered that we were able to decrypt parts of the \textbf{first} and the \text{fourth} messages. Once the latter portions of the two messages were decrypted, we were reminded of a couple of quotes from the movie, The Imitation Game, which ultimately led us to find the two messages. The two messages:\\
\inputminted{python}{decrypted-message.txt}

\answer{1-3 - Subverting Cryptography}
\begin{enumerate}
\item Insecure implementation: the implementation of a theoretically secure algorithms might have vulnerabilities
	\begin{itemize}
	\item The OpenSSL library this year was revealed to have a buffer over-read bug, Heartbleed.
	\item The NSA has inserted faulty code/backdoors in commonly used random number generators.
	\end{itemize}
	\textbf{DEFENSE: } Do not use 3rd party code. Implement everything, like random number generators and even basic algorithms, on your own.
\item Improper use: users may misuse otherwise secure software and/or algorithms
	\begin{itemize}
	\item Users may create short or weak keys
	\item Users may re-use passwords
	\end{itemize}
	\textbf{DEFENSE: } Create strong restrictions for user inputs to ensure strong key creation.
\item Hardware injection: tampering with physical devices
	\begin{itemize}
	\item Keyloggers to track keystrokes and potentially passwords
	\item Inserting backdoors into internet routers
	\end{itemize}
	\textbf{DEFENSE: } Only use devices from a trusted resource, only use devices you create yourself (i.e. create all your own routers and keyboards).
\item Government: Legal Retrieval 
	\begin{itemize}
	\item Court orders to obtain private keys (compulsion)
	\item Abusing trust commercial trust has in a body like the NSA, and just asking for the data
	\end{itemize}
	\textbf{DEFENSE: } Host data in a different country, start your own country without these rules. Don't put your trust in governmental bodies, or other bodies of "authority." Get a good lawyer. 
\item Social engineering: extracting personal information to gain useful information
	\begin{itemize}
	\item Email or phone phishing
	\end{itemize}
	\textbf{DEFENSE: } Adblock. Spam filters. Common Sense.
\item Indirect Computational Data
	\begin{itemize}
	\item Timing attacks (i.e. side channel timing attacks) that determine message based on how long it 
	\item Using metadata of a message to learn information about the messagetakes per step of computation
	\end{itemize}
	\textbf{DEFENSE: } Introduce more randomness (i.e. in length) into actual message, to make timing more uniform. Introduce randomness into metadata.
\item Coercion: bribery and corruption
	\begin{itemize}
	\item Pay NSA employees more than the government to spill secrets
	\item Give money in exchange for information
	\end{itemize}
	\textbf{DEFENSE: } Pay your employees a sufficient wage. Only hire people you trust.
\item Go after key aggregators instead of the actual message.
	\begin{itemize}
	\item Instead of trying to break the crypto, just try to steal the keys.
	\end{itemize}
	\textbf{DEFENSE: } Don't use key aggregators. Use one-time keys.

\end{enumerate}

\end{document}

