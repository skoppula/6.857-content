% 6.857 homework template
\documentclass[11pt]{article}

\newcommand{\team}{Julia Huang \\ Skanda Koppula \\ Kimberly Toy}
\newcommand{\ps}{Problem Set 3}

%\pagestyle{headings}
\usepackage{amsfonts}
\usepackage{minted}
\usepackage{graphicx}
\usepackage{amssymb}
\usepackage{amsmath}
\usepackage{url}
\usepackage{latexsym}
\setlength{\parskip}{1pc}
\setlength{\parindent}{0pt}
\setlength{\topmargin}{-3pc}
\setlength{\textheight}{9.5in}
\setlength{\oddsidemargin}{0pc}
\setlength{\evensidemargin}{0pc}
\setlength{\textwidth}{6.5in}

\newcommand{\answer}[1]{
\newpage
\noindent
\framebox{
	\vbox{
		6.857 Homework \hfill {\bf \ps} \hfill \# #1  \\ 
		\team \hfill \today
	}
}
\bigskip

}


\begin{document}

\answer{3-1 Modes of Operation}

\answer{3-1 Stream Ciphers}

\answer{3-3 AES Distinguisher}
A. We need to determine if two samples are drawn from the same distribution. For this, we use the Kolmogorov-Smirnov test that can be used to test whether the two underlying probability distributions for two one-dimensional samples are different. This is what we desire when analyzing our two samples ${x_1, x_2, x_3, ..., x_n}$ and ${y_1, y_2, y_3, ..., y_n}$. The KS test outputs a p-value, if below an appropriate threshold (e.g. 0.10), the samples are probably have different underlying distributions.

For our implementation of Kolmogorov-Smirnov, we use the \mintinline{bash}{scipy.stats} package:
\inputminted[firstline=8,lastline=18]{python}{problem3.py}

B and C. For our implementation of AES/Rijndael, we used a Python script based on an implementation by Bram Cohen: \url{http://wiki.birth-online.de/snippets/python/aes-rijndael}. We modified the script to take the number of rounds as an initialization parameter. Using this, we calculated two KS scores for each 0<r<21, where r is the number of rounds in our Rijndael: (1) the KS between the sample of distinct bytes in $AES_r=F(r,p,q)$ and $AES_{10}=F(10,p,q)$, and (2) the KS between the sample of distinct bytes in $AES_r=F(r,p,q)$ and a sample of random bytes of same size. To do this, we wrote the script on the next page. We obtained the following output:

\inputminted[firstline=26,lastline=60]{python}{problem3.py}

\end{document}

