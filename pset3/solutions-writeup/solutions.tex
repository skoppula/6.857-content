% 6.857 homework template
\documentclass[11pt]{article}

\newcommand{\team}{Julia Huang \\ Skanda Koppula \\ Kimberly Toy}
\newcommand{\ps}{Problem Set 3}

%\pagestyle{headings}
\usepackage{amsfonts}
\usepackage[table,xcdraw]{xcolor}
\usepackage{minted}
\usepackage{graphicx}
\usepackage{amssymb}
\usepackage{amsmath}
\usepackage{url}
\usepackage{latexsym}
\setlength{\parskip}{1pc}
\setlength{\parindent}{0pt}
\setlength{\topmargin}{-3pc}
\setlength{\textheight}{9.5in}
\setlength{\oddsidemargin}{0pc}
\setlength{\evensidemargin}{0pc}
\setlength{\textwidth}{6.5in}

\newcommand{\answer}[1]{
\newpage
\noindent
\framebox{
	\vbox{
		6.857 Homework \hfill {\bf \ps} \hfill \# #1  \\ 
		\team \hfill \today
	}
}
\bigskip

}


\begin{document}

\answer{3-1 Modes of Operation}

\answer{3-1 Stream Ciphers}

\answer{3-3 AES Distinguisher}
A. We need to determine if two samples are drawn from the same distribution. For this, we use the Kolmogorov-Smirnov test that can be used to test whether the two underlying probability distributions for two one-dimensional samples are different. This is what we desire when analyzing our two samples ${x_1, x_2, x_3, ..., x_n}$ and ${y_1, y_2, y_3, ..., y_n}$. The KS test outputs a p-value, if below an appropriate threshold (e.g. 0.10), the samples are probably have different underlying distributions.

By using KS, we do not have to make any assumptions about the underlying distribution of byte values after applying AES, but we sacrifice sensitivity in our test because of this.

For our implementation of Kolmogorov-Smirnov, we use the \mintinline{bash}{scipy.stats} package:
\inputminted[linenos, firstline=8,lastline=18]{python}{problem3.py}

B and C. For our implementation of AES/Rijndael, we used a Python script based on an implementation by Bram Cohen: \url{http://wiki.birth-online.de/snippets/python/aes-rijndael}. We modified the script to take the number of rounds as an initialization parameter. Using this, we calculated two KS scores for each $0<r<21$, where r is the number of rounds in our Rijndael: (1) the KS between the sample of distinct bytes in $AES_r=F(r,p,q)$ and $AES_{10}=F(10,p,q)$, and (2) the KS between the sample of distinct bytes in $AES_r=F(r,p,q)$ and a sample of random bytes of same size. To do this, we wrote the following script:

\inputminted[linenos, firstline=26,lastline=60]{python}{problem3.py}

Running this, we obtain the following KS test p-values for each value of $r$:

\begin{table}[h]
		\centering
		\begin{tabular}{lll}
				\rowcolor[HTML]{EFEFEF} 
				$r$  & KS with $AES_{10}$      & KS with random     \\
				1  & 2.59703115025e-106 & 6.54985540257e-103 \\
				2  & 6.45608663382e-36  & 5.85752036099e-34  \\
				3  & 0.610502258812     & 0.96900093708      \\
				4  & 0.341534156389     & 0.888261317501     \\
				5  & 0.288198469358     & 0.888261317501     \\
				6  & 0.199937973955     & 0.466357176296     \\
				7  & 0.0694035119046    & 0.341534156389     \\
				8  & 0.996950969554     & 0.0334029464294    \\
				9  & 0.341534156389     & 0.828373324103     \\
				10 & 1.0                & 0.466357176296     \\
				11 & 0.536650292217     & 0.536650292217     \\
				12 & 0.108571554573     & 0.341534156389     \\
				13 & 0.828373324103     & 0.759591728785     \\
				14 & 0.466357176296     & 0.288198469358     \\
				15 & 0.134165406216     & 0.828373324103     \\
				16 & 0.341534156389     & 0.828373324103     \\
				17 & 0.134165406216     & 0.401042886256     \\
				18 & 0.888261317501     & 0.341534156389     \\
				19 & 0.466357176296     & 0.536650292217     \\
				20 & 0.341534156389     & 0.828373324103    
		\end{tabular}
\end{table}

After three rounds, the p-value increases significantly to above 0.10; at that point, our test becomes ineffective in distinguishing both random bytes and $AES_{10}$ from $AES_{r}$. As a sanity check, our KS test with $AES_{10}$ had a p-value of 1.0 when $r=10$, and indeed, there should be exactly the same distribution of bytes when computing the same exact ciphertext.

\end{document}

