% 6.857 homework template
\documentclass[11pt]{article}

\newcommand{\team}{Julia Huang \\ Skanda Koppula \\ Kimberly Toy}
\newcommand{\ps}{Problem Set 3}

%\pagestyle{headings}
\usepackage{amsfonts}
\usepackage[table,xcdraw]{xcolor}
\usepackage{graphicx}
\usepackage{amssymb}
\usepackage{amsmath}
\usepackage{url}
\usepackage{latexsym}
\setlength{\parskip}{1pc}
\setlength{\parindent}{0pt}
\setlength{\topmargin}{-3pc}
\setlength{\textheight}{9.5in}
\setlength{\oddsidemargin}{0pc}
\setlength{\evensidemargin}{0pc}
\setlength{\textwidth}{6.5in}

\newcommand{\answer}[1]{
\newpage
\noindent
\framebox{
	\vbox{
		6.857 Homework \hfill {\bf \ps} \hfill \# #1  \\ 
		\team \hfill \today
	}
}
\bigskip

}

\begin{document}

\answer{3-2 Stream Ciphers}

This problem took approximately 15 man-hours. 

For convenience, the four constructions have been re-stated here:
\\A. \(x_i = f(i)\)
\\B. \(x_i = f(x_{i-1}) \oplus i\)
\\C. \(x_i = f(x_{i-1}  \oplus i)\)
\\D. \(x_i = f(x_{i-1}  \oplus i) \oplus i\)

The three constructions that have constant quality \(\theta(1)\) are A, C, and D, while construction B has a quality of \(\theta(\sqrt{2^n})\)
\\
\\To prove the \(\theta(1)\) bound on A, C, and D, we will present mappings for \(f\) and starting values for \(x_0\) that will result in all values of \(x_i\) being the same. 
\\
\\ \textbf{Construction A}: Let \(f(i)=0\) for any \(i\) and for any starting value of \(x_0\). This will trivially cause all values of \(x_i\) to equal 0. 
\\ \textbf{Construction C}: Again, let \(f(i)=0\) for any \(i\) and for any starting value of \(x_0\). Because \(x_i\) is the result of \(f\) applied to some value, this will cause all values of \(x_i\) to equal 0. 
\\ \textbf{Construction D}: Let \(f(i)=i\) and \(x_i = 0\). The first value, \(x_1\) will equal zero, as shown:
\[x_1 = f(x_0 \oplus i ) \oplus i\]
\[x_1 = f( 0 \oplus i ) \oplus i \]
\[x_1 = f(i) \oplus i =  i \oplus i = 0\]
This same logic can be used to show that all values of \(x_i\) would equal 0. 
\\
\\For \textbf{construction B}, we will demonstrate a mapping \(f\) that achieves \(\theta(\sqrt{2^n})\) quality and then show that a lower quality cannot be achieved. The idea behind this construction is to "mask" half of the bits in each number, or set those bits equal to zero.  This scheme would reduce \(2^n\) unique n-bit numbers into only \(\sqrt{2^n}\) unique numbers. 
\\
\\Given B's construction, the portion \(f(x_{i-1})\) will operate as a mask, which will be applied to \(i\). As it is difficult to create a mapping \(f\) that will consistently mask either only the first \(n/2\) bits of \(i\) or only the second \(n/2\) bits, we will create a scheme that alternately masks out the first and second halves of the bit strings. This will result in \(2* \sqrt{2^n}\) unique numbers, which is still within the \(\theta(\sqrt{2^n})\) bound. 
\\
\\Knowing that each result, \(x_i\) should have either the first \(n/2\) bits or second \(n/2\) equal to zeroes due to our masking scheme, we only need to construct a mapping \(f\) for bit strings in those forms: \(s\ ||\ n/2\ zeroes \) or \(n/2\ zeroes\ ||\ s\), where \(s\) is any bit string of length n/2.  In order to create our alternating mask, we will define the mapping f as follows:
\[f(n/2\ zeroes\ ||\ s) = s\ ||\ n/2\ zeroes\]
\[f(s\ ||\ n/2\ zeroes) = n/2\ zeroes\ ||\ (s + 1)\]
where if \(s+1\) overflows \(n/2\) bits, the top-most bit will be cut off.  Let \(x_i=0\).  
This function of \(f\) guarantees that top half of the next number \(i\) will be masked out by the portion \(s\) or the lower half by \(s+1\), and as stated earlier, this meets the \(\theta(\sqrt{2^n})\) bound. 
\\
\\ \textbf{Proof of \(\sqrt{2^n}\) lower bound:} In order to create a mapping \(f\), such that there are fewer than \(\theta(\sqrt{2^n})\) unique \(x_i\), we would need to create a scheme that masks more than \(n/2\) bits of each number \(i\). We will prove that it is impossible to create such a scheme. 

Let \(m\) be some integer greater than \(n/2\) that represents the number of bits that our scheme can mask. In order for the mask to change \(m\) bits of each \(i\) into zeroes, the mask has to replicate \(m\) bits of \(i\). 

The number of possible \(x_i\) that can be outputted is \(2^{n-m}\) because the result must have at least \(m\) zeroes. 
The number of unique masks, or outputs of \(f(x_{i-1})\), needed to implement the outlined scheme is \(2^m\). 
Because each \(x_i\) becomes an input for \(f(x_{i-1})\), we can see that this scheme would require \(f\) to map \(2^{n-m}\) inputs to \(2^m\) outputs.  As \(m > n/2\), it follows that \(2^m > 2^{n-m}\), which shows that \(f\) is an impossible mapping. 

Therefore, construction B has a quality of \(\theta(\sqrt{2^n})\). 

\end{document}

